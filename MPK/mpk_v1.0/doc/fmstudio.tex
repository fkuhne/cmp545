\documentclass[letter,12pt]{article}

\begin{document}

\title{FMSTUDIO User Manual}
\maketitle
%\tableofcontents

\section{Introduction}
\label{sec:intro}

Fmstudio is an experimental testbed that integrates the A-SBL planner
(or, alternatively, the SBL planner if compiled without the
ADAPT_COLLCHECKER option) and a simple path smoother with a plain but
powerful user interface.  To keep the code small and portable, we did
not provide fancy buttons or menus.  Instead, most commands are simply
given by pressing specific keys on the keyboard.

However, fmstudio contains some convenient methods for setting up and
storing configurations and for planning tasks.  In particular, the
user can modify the configuration by clicking on robot links and
then changing their parameters with the keyboard.  Other options include a
Cartesian interface and a target point selection method, both based on
incremental inverse kinematics computation.

After start-up, fmstudio opens a graphics window that shows the loaded
scene file (one of the files with '.iv' extension in the scenes/
subdirectory).  Furthermore, it dumps status information and a simple
menu with all possible commands in the shell window from which it was
started.  Two lists of configurations are maintained by the program:
'Database config' and 'Plan config'.  By default, the former is read
upon start-up from a database file with the same base name as the
given scene file but extension '.conf'.  The latter list ('Plan
config') can then be filled by the user to define a planning task
(consisting of two or more key configurations).  Planning tasks and
fully planned paths can be edited and saved.

In the following 'mpk home directory' will refer to the directory into
which mpk was installed (for version 1.0, this is mpk\_v1.0/).


\section{Example session}
\label{sec:example_session}

\begin{enumerate}
  
\item Start the program from the command line (make sure you are in
  the mpk home directory, e.g., mpk_v1.0/):

\begin{verbatim}
  prog/fmstudio scenes\demo1.iv 1 2
\end{verbatim}

You may skip the additional arguments 1 and 2 that define the planning
task (these are indices of those configurations in scenes/demo1.conf
that will be initially inserted as key configurations into the 'Plan
config' list; see \ref{sec:database_commands} for more details).  If
you do skip these arguments, you will have to setup the planning task
manually (see (*) in the next step).  Take a look at the menu and
status output inside the shell window from which you started fmstudio
(you may have to scroll up to see everything).

\item Click on the arrow icon at the right top corner of the FMSTUDIO
  window (or press ESC when the mouse pointer is inside the graphics
  area).  This allows you to enter keyboard commands.  In the
  following, keep the mouse pointer inside the graphics window.
  
  (*) If fmstudio was started without additional parameters 1 and 2:
  setup the planning task by pressing ENTER, then SHIFT+END, then
  ENTER again (see \ref{sec:database_commands} for more details).
  
  The most recent status output in the shell window should now contain
  the following line (somewhere between a lot of other output):
  \begin{verbatim}
  Plan config        : 1/2 * - not planned yet
  \end{verbatim}
  
\item Press F2 (this starts the planner).  Notice the running numbers
  in the shell window (the number of samples generated by the
  planner).  The planner stops when it has found a path or when it has
  reached a maximum number of samples.  When the planner has finished,
  fmstudio again dumps its menu and updated status information in the
  shell window.  Check the line that starts with 'Plan config...'
  again to see if the planner succeeded.  If not, press F2 again to
  try another run of the planner.  Repeat this step until successful.
  Once successfully planned, go to the next step. (Note: for the given
  example, the planner should succeed with the given maximum number of
  samples most times.  To change this number for more difficult
  problems, see \ref{sec:general_commands})
  
\item Press T to show the traced end-effector path.  (For the simple
  demo1.iv example with a single rigid free-flying object, the path
  traced by the origin of the object's local frame is shown.)
  
\item Press F3 to smoothe the path. The smoother dumps information for
  each smoothing step in the shell window and stops after a certain
  fixed number of steps.  F3 can be pressed repeatedly to smoothe
  more, if desired.

\item press F5 to animate the solution.

\end{enumerate}

Try other examples as well ('.iv'-files in the scenes/ subdirectory).


\section{Command line options}
\label{sec:command_line_options}

The general command line syntax is (when starting from within the mpk
home directory, e.g., mpk_v1.0):

\begin{verbatim}
  prog/fmstudio <scenefile> [-b <level>] [-db <filename>] [<idx1> <idx2> ...]
\end{verbatim}

Fmstudio requires at least one command line parameter: the name of the
scene file (e.g., scenes/demo1.iv).

The '-b' option, followed by an integer number, can be used to specify
a different level of bounding volumes for visualization (bounding
volume visualization can be switched on/off by pressing the 'B' key,
see \ref{sec:general_commands}).  By default, the top-level BVs of
each object are shown.  Only one level of BVs can be shown (too
cluttered otherwise).

The '-db' option, followed by a file name, can be used to specify the
name of the configuration database explicitly.  By default, fmstudio
reads the configuration database ('Database config', see
\ref{sec:database_commands}) from a file with the same base as the
scene file but with '.conf' extension - provided that this file
exists.  With this option, an arbitrary file name can be given.

An optional sequence [<idx1> <idx2> ...] of integer indices (of
arbitrary length) can be given to define an initial task.  That is,
the configurations with the given indices (from the 'Database config'
list) will be inserted into the 'Plan config' list.  Note that indices
start with 1.  If no indices are given, the task needs to setup
manually (see \ref{sec:database_commands}).


\section{Keyboard commands and mouse interface}
\label{sec:commands}

Most commands are given by pressing a particular key while the mouse
pointer is inside the graphics window.  Note that in order to accept
keyboard commands, the graphics window must be in {\bf input/selection
  mode} (in this mode, the mouse pointer is shown as an arrow).  To
switch to input/selection mode, click on the arrow symbol at the top
right corner of the graphics window.  Alternatively, pressing ESC
while the mouse is inside the graphics area toggles between
input/selection mode and {\bf examination mode}.  In examination mode,
the entire scene can be rotated/moved with the mouse (this is handled
by Open Inventor).  All keyboard commands and mouse actions that are
handled by fmstudio require input/selection mode.


\subsection{General keyboard commands}
\label{sec:general_commands}

Most commands are given by pressing a single key on the keyboard while
the mouse pointer is inside the graphics window.  Make sure that the
graphics window is in input/selection mode.  (If keyboard commands
seem not to work, either click on the arrow in the top right corner of
the graphics window or press ESC while the mouse pointer is in the
graphics window.)  For some commands below, the key to be pressed is
in parentheses.

\begin{description}
  
\item[ESC] Switch between camera control (examination mode) and
  keyboard/mouse control (input/selection mode).  In examination mode,
  the mouse can be used to rotate/move/zoom the entire scene.
  Keyboard/mouse mode is necessary to invoke all other commands below.
  
\item[F1] Toggle parameter/Cartesian/target mode (see
  \ref{sec:mouse}).
  
\item[F2] Invokes the planner on all successive pairs of key
  configurations in the 'Plan config' list (see
  \ref{sec:database_commands}).  Upon success, the planner inserts
  additional, non-key configurations into the 'Plan config' list to
  make it a collision-free path.
   
\item[F3] Smoothes the path in 'Plan config'.  The smoother does not
  remove key configurations.

\item[F4] Plan + smoothe (equals pressing F2 and then F3)

\item[F5] Start/stop animation of the path in 'Plan config'.

\item[F6] Reset animation (move to first configuration in 'Plan config')
  
\item[F11] Freeze/unfreeze current link.  This affects the currently
  selected (picked) joint (see \ref{sec:mouse}).
  
\item[F12] Freeze/unfreeze currently selected (picked) robot (see
  \ref{sec:mouse}).
 
\item[SPACE] Change the value of the control step size.  This value
  affects the robot's position control.  Pressing SPACE iterates
  through a set of possible values, in powers of 10.

\item[NUM5] Identical to SPACE.
  
\item[(D)elta] Change the collision checker delta-value.  This is a
  minimum workspace clearance required to bound the running time of
  the adaptive collision checker.  Also, if compiled with
  DO_TOLERANCE_TEST, this value is used in the tolerance test of the
  simple (constant epsilon) segment checker.  After pressing 'D', move
  the mouse pointer to the shell window and enter a new value.
  
\item[(E)psilon] (may not appear if compiled with ADAPT_COLLCHECKER).
  Change the segment collision checker epsilon value.  This is the
  (constant) c-space resolution for the simple segment checker.  After
  pressing E, move the mouse pointer to the shell window and enter a
  new value.
  
\item[(R)ho] Change sampling window size of the planner.  This is the
  size of the window (for each DOF) within which the local neighbors
  of a milestone are sampled.  After pressing 'R', move the mouse
  pointer to the shell window and enter a new value.
  
\item[(M)ax. planner iter.]  Change the maximum number of planner
  iterations.  When reaching this number without having found a path,
  the planner gives up.  To change the value, press 'M', move the mouse
  pointer to the shell window and enter a new value.  The default
  value works for many simple planning problems.  For difficult
  planning problems, one may need to increase this parameter to a
  higher value.
  
\item[Sm(O)othe steps] Change the number of attempted shortcut steps
  performed by the path smoother.  This parameter does only affect the
  smoother.  After pressing 'O', move the mouse pointer to the shell
  window and enter a new value.
  
\item[(A)nimsteps (max.)]  Change the animation speed (higher values
  result in slower animation).  After pressing 'A', move the mouse
  pointer to the shell window and enter a new value.
  
\item[Sample (F)ree random config] Sample a random free configuration
  for the current configuration shown in the graphics window.  Note:
  this changes the configuration that is displayed in the graphics
  window but does not affect the 'Plan config' and 'Database config'
  lists.
  
\item[(B)ounding boxes (OBB/RSS/off)] Toggle the displayed bounding
  volumes.  By default, the top-level bounding volumes of all
  robots/obstacles are shown only.  Use the '-b' option (see
  \ref{sec:command_line_options}) to show BVs of a different level
  instead.  Note that RSSs are not shown as RSSs.  Instead, their
  smallest enclosing OBBs are shown.  Pressing 'B' repeatedly several
  times shows OBBs, then slightly bigger OBBs (those enclosing the
  RSSs), next no BVs, then again OBBs and so on.
  
\item[(T)raced end-effector path on/off] Show/hide the paths traced by
  all robots' trace points (see documentation on how to create new
  robots in doc/mpk_new/index.html).
  
\item[(I)mage snapshot...] Save a snapshot of the current scene (Unix
  version only).  This requires the ImageMagick program 'convert' be
  installed.  After pressing 'I', move the mouse pointer to the shell
  window and enter a file name, e.g., 'test.jpg'.
  
\item[(V)ideo file name] Use this command to enter a file name (e.g.,
  'test.jpg') if you would like to create a sequence of snapshots
  during animation (useful for creating videos).  Sequential numbers
  will be added to distinguish the files.  Set the file name to '-' to
  disable dumping video frames.  Note: like the 'Image snapshot'
  option above, this option requires the 'convert' program.
  Furthermore, it will slow down the animation considerably (for each
  frame, an image file will be saved).  To enter the video file name,
  press 'V' and then move the mouse pointer inside the shell window.
  
\item[(Q)uit] Exits the program.  The current configuration database
  'Database config' is saved in a file.  By default, this file has
  the same name base as the scene file plus a '.conf' extension.  An
  alternative name can be specified using the '-db' option.

\end{description}



\subsection{Configuration database and task commands}
\label{sec:database_commands}

Two lists of configurations are maintained by fmstudio: the 'Plan
config' list and the 'Database config' list.

The latter list ({\bf 'Database config'}) is meant as a repository of
useful configurations.  It is typically stored in a file with the same
name base as the scene file but ending in '.conf' instead of '.iv'.
It is read right after starting the fmstudio and written back to the
same file when the program exits normally.  (Use the '-db' option, see
\ref{sec:command_line_options}, to specify a different file name.)

In contrast, {\bf 'Plan config'} contains a task, and possibly further
configurations that define an entire plan (that is, a solved task).
(Use a sequence of indices at the command line, see
\ref{sec:command_line_options}, to fill this list at start-up of
fmstudio.)  A 'task' is defined as an arbitrary sequence of
collision-free, so-called {\bf 'key-configurations'}.  A 'plan'
contains additional (non-key) configurations such that connecting two
successive configurations by straight line segments yields a
collision-free path.  The planner tries to insert proper non-key
configurations into a task to turn it into a plan.

Thus, 'Database config' is a conglomeration, most likely without
significant order of the configurations, and is automatically saved at
proper exit of fmstudio.  In contrast, the order of the configurations
in 'Plan config' does matter.  'Plan config' can be seen as something
like the 'working memory' of fmstudio and needs to be saved explicitly
if desired (see below).  Both lists can be edited by the user.  To
this end, each list has a pointer associated with it that indicates
the current position, similarly to a cursor position, in its list.
The status output in the shell window shows the sizes ('plan-total'
and 'db-total') of both lists and the respective current 'cursor'
positions 'plan-curr' and 'db-curr':

\begin{verbatim}
  Plan config     : <plan-curr>/<plan-total> * - not planned yet
  Database config : <db-curr>/<db-total> (database file: <filename>)
\end{verbatim}

An asterisk '*' in the 'Plan config' line shows if the configuration
at the 'plan-curr' position in 'Plan config' is a key configuration
(only key-configurations are considered when the planner is run; this
allows re-starting the planner several times in a row on the same
task).  The key status flag can be changed by the user (see below).
Furthermore, the 'Plan config' line indicates the planner status.

Similarly, the 'Database config' line shows the current position in
the 'Database config' list and the total number of stored
configurations.  It also shows the name of the file from which the
database was read at program start and to which it will be written
back when the program exits normally.

The following keyboard commands can be used to edit and perform other
operations on these two lists:

\begin{description}
  
\item[INS(+)] Insert the current configuration (the one shown in the
  graphics window) as a key configuration at the current position in
  the 'Plan config' list.  The current position remains fixed, the
  number of configs is increased by one.
  
\item[DEL(+)] Delete configuration from 'Plan config' list at current
  position in 'Plan config'.
  
\item[ENTER(+)] Append current configuration (the one shown in the
  graphics window) at the end of the 'Plan config' list.
  
\item[HOME(+)] Step to the previous configuration in the 'Plan config'
  list and display this configuration in the graphics window.
  
\item[END(+)] Step to the next configuration in the 'Plan config' list
  and display this configuration in the graphics window.
  
\item[+SHIFT] Pressing SHIFT together with the commands marked with
  '(+)' applies these commands to the 'Database config' list instead of
  the 'Plan config' list.

\item[CTRL+SHIFT+DEL] Clear the entire 'Plan config' list.
  
\item[(K)ey/unkey current configuration in plan] Toggle the key status
  flag of the current configuration in the 'Plan config' list. (See
  top of section for definition of the key status flag.
  
\item[(G)oto database config \#...]  Sets the current position in the
  'Database config' list to a user-specified number.  After pressing
  'G', move the mouse pointer to the shell window and enter the
  desired index (note that indices start with 1).  The chosen
  configuration is then shown in the graphics window and the current
  position in the 'Database config' list is updated correspondingly in
  the shell window.
  
\item[(L)oad task/path...]  Load 'Plan config' list from file (the
  current list in memory will be overwritten).  After pressing 'L',
  move the mouse pointer to the shell window and enter the name of a
  valid task/path file (e.g., one saved previously using the 'S' command
  below).
  
\item[(S)ave task/path...] Save 'Plan config' list to file.  After
  pressing 'S', move the mouse pointer to the shell window and enter a
  file name.  Note: an existing file with the same name will be
  overwritten without any warnings!
 
\end{description}


\subsection{Mouse interface}
\label{sec:mouse}

When the graphics window is in examination mode (see
\ref{sec:commands}), the mouse can be used to rotate/move the entire
scene.  This is handled by Open Inventor.

When in input/selection mode, the mouse can be used to select robots
and individual robot links (use the left mouse button).  The status
output in the shell window shows which robot and joint/link is
currently selected:

\begin{verbatim}
  Picked joint & link: <picked-joint>
  Picked robot: <picked-rob>
\end{verbatim}

Once a robot link has been selected by clicking on it, the
configuration can be modified by changing individual DOFs (parameter
mode), by dragging links with a 'rubber' band (Cartesian mode) or by
clicking on target points for a robot's end-effector (target mode):

\begin{description}
  
\item[Parameter mode] In parameter mode (default), the user can click
  on a robot link and then use the cursor keys to increase/decrease
  the corresponding DOF parameter (in steps), typically a joint angle.
  If more than one DOF affect the picked link (e.g., a free-flying
  robot), then press SHIFT and use the right mouse button to switch
  through these DOFs (observe how the status output next to 'Picked
  joint \& link' in the shell window changes).  To change the step
  size, use the SPACE (or NUM5) command \ref{sec:general_commands}.
  
\item[Cartesian mode] In Cartesian mode, a cross cursor appears in the
  graphics window that can be moved in space with the cursor keys
  (plus PGUP/PGDN).  See SPACE (or NUM5) command
  \ref{sec:general_commands} to change the step size.  The user can
  then click on a robot link to attach a rubber band between the
  selected point on the robot and the center of the cross cursor.  The
  robot then follows the cross cursor.  This allows to move the
  end-effector in Cartesian space, e.g., along a line in workspace.
  Note that the Cartesian mode works only if the function
  mpk_optimize() in file mpk_opt.cpp is implemented (we cannot include
  this function in our package, for copyright reasons, since the
  implementation that we use is taken from the book 'Numerical Recipes
  in C').
  
\item[Target mode] In target mode, the user first chooses a robot by
  clicking on an arbitrary link.  Once a robot is selected, clicking
  on an obstacle moves the robot so that its trace point comes closer
  to the selected point on the obstacle.  The 'trace point' is usually
  a point at the end-effector (see documentation in file
  doc/mpk_new/index.html on how to define new robots).  Since this is
  an inverse-kinematics problem, the solution may not be unique or not
  exist at all.  Fmstudio uses an incremental distance minimization
  approach, so several clicks on the obstacle may be necessary to move
  the robot to its final location.  Like the Cartesian mode, the
  target mode works only if the mpk_optimize() function in the file
  mpk_opt.cpp is implemented (which it is not by default).

\end{description}

To switch through these modes, press F1 (see
\ref{sec:general_commands}).  Note that for Cartesian/target modes,
the function mpk_optimize() must be implemented (we use code from the
book 'Numerical Recipes in C' which we cannot include for copyright
reasons).


\section{Collision sets}

The status output in the shell window tells how many pairs of rigid
objects are considered for the collision test and if the displayed
configuration contains a collision:

\begin{verbatim}
  Colliding pair: <collpair>
  Collision set size: <collset-sz> (<num-pruned> pairs pruned)
\end{verbatim}

The collision checker is always active to tell if the configuration in
the graphics window is in collision or not (see output 'Colliding
pair: ...').  If there are collisions of more than a single pair of
objects, an arbitrary colliding pair is reported in the status output
in the shell window.  If the minimum workspace clearance delta is set
to a value >0, collisions may be reported although there are no real
intersections.  The output 'Collision set size: ...' tells how many
pairs of rigid bodies are actually checked for intersection
('collset-sz'), and how many pairs of objects were not included in the
collision set because they cannot possibly collide ('num-pruned').


\section{Further information}

See documentation in doc/mpk_new/index.html further information
on how to:

\begin{itemize}

\item create new environments / planning problems

\item create new robots

\end{itemize}



\end{document}
