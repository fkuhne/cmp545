\documentclass[letter,12pt]{article}

\begin{document}

\title{Creating new scenes and robot types in MPK}
\maketitle
%\tableofcontents


\section{Introduction}

This document describes how to create new planning scenarios (scene
files) and how to extend MPK with new robot types.  

In principle, both can be done without re-compiling any source-code.
However, if a new robot type requires an unusual parameterization, some
re-compiling may be necessary (see hard-coded robots).


\section{Scene definition}
\label{sec:scenes}

The MPK scene file format is a simple extension of the Open Inventor
scene file format.  (For details on the Inventor file format, see an
Open Inventor manual or tutorial.)  Scene files have an '.iv'
extension and are typically in the scenes/ subdirectory.  MPK extends
the Inventor file format by three node types: {\bf mpkObstacle}, {\bf
  mpkRobot} and {\bf mpkIncludeFile} (for details, see below).  These
are derived from the Inventor node type SoSeparator and can thus have
the same attributes as a SoSeparator, in addition to their own
specific attributes that are described below.

Before creating a new scene file, it can be helpful to examine one of
the existing scene files.  E.g., take a look at 'scenes/demo1.iv'.
One can add an arbitrary number of robots (nodes of type 'mpkRobot')
to the scene file, each with its own 'scaleFactor' and 'translation'
attributes, in the same way as the one robot that is included in the
'demo1.iv' file.  See \ref{sec:robots} on which robots are available
and how to define new robots.

Besides robots, one can include one or more files (nodes of type
'mpkIncludeFile') containing obstacles, like the wall file 'wall.iv'
in 'demo1.iv'.  Obstacles can also be defined directly using nodes of
type 'mpkObstacle' (e.g., see file 'scenes/mov\_puma\_vwbody.iv').

The following descriptions summarize the available MPK node types:

\begin{description}
  
\item[mpkObstacle] Adds a static obstacle to the scene.
  
  {\bf Example:} The following example defines an obstacle named 'box'
  that is a cube of side lengths (1, 2, 3).  Note that Cube is an
  Inventor node type. 
\begin{verbatim}
DEF box mpkObstacle {
  DEF __triangulate__ Cube {
    width 1
    height 2
    depth 3
  }
}
\end{verbatim}
  
  The 'DEF \_\_triangulate\_\_' tag is required to tell MPK that the
  triangles of the Cube node are part of the collision model.
  Similarly, one could add other Inventor models inside the same
  mpkObstacle node.  Without the 'DEF \_\_triangulate\_\_' tag, the
  models would be displayed but not be checked for collisions.  Note
  that all tagged models within the same mpkObstacle node are merged
  into a single PQP hierarchy for collision checking.
  
  In the example, the 'DEF box' definition can be omitted.  However,
  it is recommended to give names to obstacles to identify which
  obstacles are involved in collisions.

\item[mpkRobot]  Adds a robot to the scene.
  
  {\bf Example 1:} In the following example, a robot named 'puma1' is
  inserted into the scene.  Its kinematics definition is read from the
  'puma500\_torch.rob' robot definition file in the 'robots/'
  subdirectory.  The robot is scaled, translated and rotated before it
  is inserted into the scene.
\begin{verbatim}
DEF puma1 mpkRobot {
  fileName "puma500_torch.rob"
  scaleFactor 2.2
  translation 0 -0.01 0
  rotation 0 -1 0 1.6
}
\end{verbatim}
  The rows with scaleFactor, translation and rotation can be omitted.
  The rotation parameters are x,y and z of the rotation axis followed
  by the rotation angle (in radians).
  
  List the '.rob'-files in the 'robots/' subdirectory to see which
  robot types are currently available.
  
  {\bf Example 2:} In this example, a hard-coded robot of type
  DEMOROBOT is inserted.  Hard-coded robots have the advantage that
  they can realize arbitrary parameterizations (see files
  'robots/mpkDemoRobot.H' and 'robots/mpkDemoRobot.cpp').
\begin{verbatim}
DEF robot mpkRobot {
  robotType DEMOROBOT
  scaleFactor 1.5
  translation 0 1 0
  rotation 0 1 0 3.2
}
\end{verbatim}
  
  Check the file 'basic/mpk\_robot\_type.h' and the '.cpp'-files in the
  'robots/'-subdirectory to see which hard-coded robot types are
  currently available.
  
  For information on how to create your own types of robots, see
  \ref{sec:robots}.


\item[mpkIncludeFile] Includes a file with a complete scene
  definition.

{\bf Syntax:}
\begin{verbatim}
DEF <identifier> mpkIncludeFile {
  name "<filename>"
}
\end{verbatim}

The part 'DEF <identifier>' can be omitted.  However, this may result
in ambiguous object identifiers if the same file is included several
times.  To avoid ambiguities, define a different name for each
mpkIncludeFile node that loads the same file.

\end{description}


\section{Robot definition}
\label{sec:robots}

In MPK, robots are defined as kinematic trees.  A node in such a
kinematic tree is a 'joint' structure, which typically defines a
rotation around an axis or a translation along a vector.  The
transform can be parameterized but need not.  A 'joint' node can
optionally have an associated link model.  For example, a kinematic
chain of three translational and three rotational joints can be used
to define a rigid free-flying object.  In this case, only the last
joint would have a 'link' model, in this case the free-flying object
(for visualization and collision checking).

MPK offers two possibilities to define the kinematic tree of a robot:

\begin{description}
  
\item[Robot files (.rob)] This is the recommended approach for
  defining robot kinematics.  Robots can be added and removed without
  recompiling any code.  However, joints types are limited to
  prismatic and revolute.  For more details, see \ref{sec:rob_file}.
  
\item[Hard-coded robots (.H, .cpp)] This approach consists of defining
  the kinematics description in source code files.  The advantage is
  that arbitrary kinematics and parameterizations can be defined.  The
  disadvantage is that the some code needs to be recompiled whenever a
  new robot type is added.  For more details, see
  \ref{sec:rob_hardcoded}.

\end{description}

\subsection{Robot definition files (.rob)}
\label{sec:rob_file}

A robot definition file ('.rob' file in the 'robots/'-subdirectory)
defines the kinematics of a particular robot type, that is, the
structure of the kinematic tree of a single robot of this type.  (See,
e.g., 'robots/puma500.rob'.) The first line in the file must be of the
form
\begin{verbatim}
  #MPK v<num> robot
\end{verbatim}
where <num> is the current MPK version number, e.g., 1.0.  The rest of
the file consists of a collection of nodes of types {\bf joint}, {\bf
  selfcoll} and {\bf param}.

Although these node structures in the robot definition files are
syntactically similar to OpenInventor nodes, the two should not be
confused.  Unlike MPK nodes in the scene definition files (see
\ref{sec:scenes}), nodes in a robot definition file are not derived
from OpenInventor nodes.  Instead, a simple specialized parser is used
to read '.rob'-files (see file 'basic/mpkBaseRobot.cpp').

A robot definition file typically contains several 'joint' nodes but
at most one 'param' and 'selfcoll' node each.  (Although an arbitrary
number of 'param' or 'selfcoll' nodes is allowed, in most cases, it
makes sense to define at most one of each.)

The following list describes the details of each node type in the
robot definition file.

\begin{description}
  
\item[joint] A 'joint' structure defines a node in the kinematic tree
  of the robot.

{\bf Syntax:}
\begin{verbatim}
joint <name> {
  <field-type1> <field-params1>
  <field-type2> <field-params2>
  ...
}
\end{verbatim}
  
  where <name> is a user-defined name for the joint, <field-type> is
  one of the types listed below, and <field-params> is a correct set of
  parameters as described for each of the types below.
  
  A 'joint' node can optionally include the following fields
  (of the form <field-type> <field-params>):

\begin{description}
  
\item[parent <parent-name>] Attach the surrounding 'joint' node as a
  child to the 'joint' node with name <parent-name>.  If this field is
  omitted, then the 'joint' node defines the root of a new kinematic
  tree that is fixed in the scene coordinate system.  Note that more
  than one such tree is possible within a '.rob'-file.  The transform
  of a 'joint' node is then the chain of transforms of all nodes on
  the path back to the (unique) root of the tree.
  
\item[ConstTransl <x> <y> <z>] Add a constant (local) translation to
  the node by the vector (<x>, <y>, <z>).
  
\item[ConstRot <ax-x> <ax-y> <ax-z> <ang>] Add a constant (local)
  rotation to the node by the angle <ang> (in radians) around the axis
  (<ax-x>, <ax-y>, <ax-z>).  The axis is assumed to go through the
  origin of the (local) coordinate system.
  
\item[Transl1 <x> <y> <z> <min> <max>] Add a (local) parameterized
  translation to the node along the vector (t*(<max> - <min>) + <min>)
  * (<x>, <y>, <z>) where t is the parameter value in [0,1].  The
  field 'param' (see below) must be added to specify the index of the
  parameter (default 0).
  
\item[Rot1 <ax-x> <ax-y> <ax-z> <min> <max>] Add a (local)
  parameterized rotation to the node by the angle t*(<max> - <min>) +
  <min> (in radians) around the axis (<ax-x>, <ax-y>, <ax-z>) where t
  is a normalized parameter in [0,1].  The axis is assumed to go
  through the origin of the (local) coordinate system.  The field
  'param' (see below) must be added to specify the index of the
  parameter (default 0).
  
\item[ConstTransl\_Rot1 <x> <y> <z> <ax-x> <ax-y> <ax-z> <min> <max>]
  Add a (local) concatenation of a parameterized rotation by the angle
  t*(<max> - <min>) + <min> (in radians) around the axis (<ax-x>,
  <ax-y>, <ax-z>), followed by a constant translation by (<x>, <y>,
  <z>) to the node.  The parameter t is normalized in the interval
  [0,1].  The axis of rotation is assumed to go through the origin of
  the (local) coordinate system.  The field 'param' (see below) must
  be added to specify the index of the parameter (default 0).
  
\item[param <num>] Specifies the index (in the configuration vector of
  the robot) of the parameter that influences a transform of type
  'Transl1', 'Rot1' or 'ConstTransl\_Rot1' (see above).  Note that at
  most one such transform can be given per 'joint' node.
  
\item[model0 <iv-fname>] Read a collision model from the file
  <iv-fname> (in OpenInventor format).  This field is not obligatory.
  If no collision model is given, the 'joint' node is considered for
  kinematics computations only but not for collision checking (see,
  e.g., file 'robots/freeflyingL.rob').
  
\item[model1 <iv-fname>] Read a second collision model.  Some
  collision checking functions of the MPK library allow specifying
  which collision model to use.  The second model could be a model of
  smaller complexity that includes the first model 'model0'.  While
  'model0' is always used for visualization, using another (simpler)
  model for collision checking may result in a faster collision check.
  Using a second collision model may be helpful for other purposes as
  well.
  
\item[coll [TRUE | FALSE]] Default: TRUE (equals omitting the 'coll'
  field altogether).  If this field is given and set to FALSE, then
  the collision models are used for visualization only but not checked
  for any collisions.
  
\item[tracePoint] If specified, a 'trace point' is added to the
  'joint' node.  A trace point follows a (static) curve segment (which
  is, in general, non-linear) in space when the robot moves from one
  configuration to another.  The GUI of MPK can display such curves of
  selected trace points to illustrate a robot's path.  The trace point
  is by default at the origin of the local coordinate system.  To
  realize another location, define an extra 'joint' node for the trace
  point that does not contain a collision model.

\end{description}

Note that multiple lines with constant transforms can be given within
a single 'joint' node.  In this case, all of them will be concatenated
in the given order into a single constant transform.  However, only a
single parameterized transform can be given.  To realize a chain of
parameterized transforms that affects a single collision model, use a
chain of 'joint' nodes, each with a single parameterized transform and
without 'model0' or 'model1' entries, except the last one.

{\bf Example 1:} The following example (taken from
'robots/puma500.rob') defines a 'joint' node with name 'Wrist3' that
is attached to the 'joint' node named 'Wrist2'.  It can rotate around
the (local) axis (0, -1, 0) driven by parameter 5 within the angular
bounds (-2.61, 2.61).  A single geometric model is read from file
'Puma500/puma_6.iv' and used for both visualization and collision
checking.
\begin{verbatim}
joint Wrist3 {
  parent Wrist2
  Rot1 0 -1 0 -2.61 2.61
  param 5
  model0 "Puma500/puma_6.iv"
}
\end{verbatim}

{\bf Example 2:} This example defines a 'trace point' that moves
together with the frame of 'joint' node 'Wrist3' but is translated by
a (constant) vector (0, 0.09, 0).
\begin{verbatim}
joint TracePoint {
  parent Wrist3
  ConstTransl 0 0.09 0
  tracePoint
}
\end{verbatim}


\item[selfcoll] A 'selfcoll' structure lists all pairs of joints whose
  link models should be checked for 'self-'collisions, that is,
  collisions of links of the same robot.  If this structure is not
  given, then no self-collisions are checked for the robot.

{\bf Syntax:}
\begin{verbatim}
selfcoll {
  <name1-a> <name1-b> 
  <name2-a> <name2-b> 
  ...
}
\end{verbatim}
where, in each row, <name-a> and <name-b> are the names of two
different 'joint' structures (see above) of the same robot.  Note that
both of these referenced 'joint' nodes should have a collision model
defined.


\item[param] The 'param' structure contains information about the
  individual parameters of the robot.  Since a parameter can affect
  more than one joint, it would not make sense to store this
  information within each individual 'joint' structure.

  {\bf Syntax:}
\begin{verbatim}
param {
  <param1-idx> { <opt1-a> <opt1-b> ... }
  <param2-idx> { <opt2-a> <opt2-b> ... }
  ...
}
\end{verbatim}
  where <param-idx> is the index of a parameter in the robots
  configuration vector (note that these indices start with 0), and
  <opt-a> <opt-b> ... are options that will affect the indexed
  parameter.  The following options are available:

\begin{description}
  
\item[cyclic] Indicates that the corresponding parameter is cyclic,
  i.e., it should 'wrap around' when it reaches a boundary of the
  normalized interval [0,1].  Default: off
  
\item[passive] Indicates that the corresponding parameter should not
  be modified during planning.  This can be used, e.g., for gripper
  DOFs to prevent the planner from creating a path on which the
  gripper randomly opens and closes.  However, the animation does
  interpolate the corresponding parameter such that the gripper can
  open and close during a motion sequence that consists of several key
  configurations (see 'doc/fmstudio/index.html' for a definition of
  'key configurations').
 
\item[weight <val>] Assigns the weight <val> to the parameter.
  Weighted parameters can be useful in certain circumstances, e.g.,
  for defining weighted norms on configurations.
  
\item[default <val>] Assigns the default value <val> (must be in
  [0,1]) to the parameter.  Unless this option is specified, the
  default value for all parameters is 0.5.

\end{description}


\end{description}



\subsection{Hard-coded robots}
\label{sec:rob_hardcoded}

Hard-coded robots offer the possibility to define arbitrary
parameterizations, but adding a new hard-coded robot requires
recompiling the MPK library and all programs that use it.  By default,
only two hard-coded robots are defined: The type DEMOROBOT (see
mpkDemoRobot.cpp,.H) demonstrates how arbitrary parameterizations can
be realized.  The other type, FREEFLYINGOBJECT (also see
mpkFreeFlyingObject.cpp,.H), is to make definition of free-flying
objects more comfortable: compare the two alternatives shown in the
file 'scenes/demo1.iv'.  Using the hard-coded type FREEFLYINGOBJECT, it
is only necessary to specify the file name of the collision model
('FreeFlyingObjects/rod.iv' in the example).  With the other
alternative, the '.rob' file must include both the kinematics
description and the name of the collision model (see
'robots/freeflyingL.rob').

To add a new hard-coded robot type 'mpkNewRob' to the MPK library,
perform the following steps (replace 'NewRob' by the name you would
like to give the robot):

\begin{enumerate}
  
\item Create a subdirectory ivmodels/NewRob and copy all inventor link
  models of the new robot into this directory.
 
\item Create files robots/mpkNewRob.cpp and robots/mpkNewRob.H (cf
  mpkDemoRobot.cpp and mpkDemoRobot.H).  In the robot definition, make
  sure that you refer to the link models that you copied into the
  'ivmodels/NewRob' directory just before.

\item Edit the file basic/mpk\_robot\_type.h and add:
  \begin{itemize}
  \item \#include "mpkNewRob.H"
  \item MPKNEWROB to 'mpk\_robot\_type'
  \end{itemize}
 
\item Edit file basic/Robot.cpp and add:
\begin{itemize}
\item SO\_NODE\_DEFINE\_ENUM\_VALUE(mpk\_robot\_type, MPKNEWROB) to
  the mpkRobot constructor
\item a 'case MPKNEWROB' for the new robot type to the switch
  statement in Robot::init()
\end{itemize}
 
\item Unix: In the 'robots/Makefile', add mpkNewRob.cpp to the SRC
  list and then call 'make' from the MPK home directory.  Windows: Add
  mpkNewRob.cpp to the current visual studio project and rebuild it.
  
\end{enumerate}

\end{document}






